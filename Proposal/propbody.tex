
% Draft #1 (final?)

\vfil

\centerline{\Large Computer Science Project Proposal}
\vspace{0.4in}
\centerline{\Large Constructing 3D models from image sequences}
\vspace{0.4in}
\centerline{\large I. G. Sorescu, Newnham College}
\vspace{0.3in}
\centerline{\large Originator: I. G. Sorescu}
\vspace{0.3in}
\centerline{\large 13$^{th}$ October 2014}

\vfil


\noindent
{\bf Project Supervisor:} K. Palyutina
\vspace{0.2in}

\noindent
{\bf Director of Studies:} Dr J. K. Fawcett
\vspace{0.2in}
\noindent
 
\noindent
{\bf Project Overseers:} Dr~A.~Rice  \& Dr~T.~G.~Griffin


% Main document

\section*{Introduction and Description of the Work}

The aim of this project is to extract a 3D model of a small object from a sequence of plain images and then to export the model into a 3D geometry format such as PLY. The resulting file can then be rendered and processed by MeshLab (or similar).  \\
The main focus of the project is to obtain a wireframe model, leaving shading and texture mapping as a possible extension. \\
The inputted set of pictures must be obtained using the following technique [add an image to represent this]: 
{\em Define an "origin" on a perfectly horizontal table and place the camera there facing a white background. Place an object of max height X at a distance Y from the origin. Take the first picture. Rotate the object 30 degrees (counter clockwise as seen from camera) and take the second picture ... until 12 pictures are taken. Some other details to work out such as focal distance. Update pictures in the order in which they were taken. Assume well-lit room and no major light differences between pictures in the same set.} \\
Extension: implement self-calibration such that in theory the project will work on any set of pictures as long as large overlap.

\section*{Starting Point}
Some experience with C/C++.
IB maths and graphics.

\section*{Resources Required}
Camera(add detailed description), my own computer + backup plans

\section*{Work to be done}
For core bits (strictly sufficient to satisfy the success criterion):
\begin{itemize}
\item detect edges in the 12 pictures
\item computing a dense set of correspondences between neighbouring images
{\emph stereo rectification, stereo matching, dense depth map}
\item reconstructing the 3D object shape
{\emph overlying a 2D triangular mesh on top of one of the images to bild corresponding 3D mesh by wrapping vertives of the trianglesin 3D space by using depth maps OR volumetric depth map integration, Kalman filter}
\end{itemize}
Extensions:
\begin{itemize}
\item self-calibration: computing the geometric relation between neighbouring images, estimating the motion and calibration of the camera
\item {\emph search for texture mapping techniques}
\end{itemize}

\section*{Success Criterion}
 A system that can successfully construct the wireframe of a fairly regular object of maximum height X [more specific?] from pictures taken as described in introduction. \\
 Might be helpful (but daunting) to compare results to those generated by ARC3D.
 
\section*{Possible Extensions}
\begin{itemize}
\item add shadowing to model
\item add texture to model
\item self-calibration
\end{itemize}

\section*{[unbelievably unrealistic] Timetable}

Planned starting date is 09/10/2014.

\begin{enumerate}

\item {\bf Slot 0: 9$^{th}$ Oct - 24$^{th}$ Oct}
	\begin{itemize}
		\item Read relevant literature and plan the project.
	\end{itemize}
	{\bf Milestone:} Submit proposal
\item {\bf Slot 1: 25$^{th}$ Oct - 14$^{th}$ Nov}
	\begin{itemize}
		\item Gain deeper understanding of the techniques
		\item Plan
	\end{itemize}
	{\bf Milestone:} 
\item {\bf Slot 2: 15$^{th}$ Nov  - 5$^{th}$ Dec}
	\begin{itemize}
		\item Start implementation 
	\end{itemize}
	{\bf Milestone: finish implementing edge-detector} 
\item {\bf Slot 3: 6$^{th}$ Dec -  26$^{th}$ Dec}
	\begin{itemize}
		\item Implement dense set stuff
	\end{itemize}
	{\bf Milestone: finish implementing dense set stuff} 
\item {\bf Slot 4: 27$^{th}$ Dec - 16$^{th}$ Jan}
	\begin{itemize}
		\item Implement mesh stuff
	\end{itemize}
	{\bf Milestone: Finish implementation} 
\item {\bf Slot 5: 17$^{th}$ Jan  - 30$^{th}$ Jan}
	\begin{itemize}
		\item Buffer time: catch up or start doing the extensions
		\item Write progress report
	\end{itemize}
	{\bf Milestone: Submit progress report} 
\item {\bf Slot 6: 31$^{st}$ Jan - 20 $^{th}$ Feb}
	\begin{itemize}
		\item Debug and test
	\end{itemize}
	{\bf Milestone: write and pass tests} 
\item {\bf Slot 7: 21$^{st}$ Feb - 13$^{th}$ Mar}
	\begin{itemize}
		\item Catch-up time or extensions
	\end{itemize}
	{\bf Milestone:} 
\item {\bf Slot 8: 14$^{th}$ Mar - 3$^{rd}$ Apr}
	\begin{itemize}
		\item Analysis
	\end{itemize}
	{\bf Milestone:} 
\item {\bf Slot 9: 4$^{th}$ Apr - 17$^{th}$ Apr}
	\begin{itemize}
		\item Start writing dissertation
	\end{itemize}
	{\bf Milestone:}
\item {\bf Slot 10: 18$^{th}$ Apr - 8$^{th}$ May}
	\begin{itemize}
		\item Finish writing dissertation
	\end{itemize}
	{\bf Milestone:}	 
\item {\bf Slot 11: 9$^{th}$May  - 15$^{th}$May}
	\begin{itemize}
		\item Safety slot
	\end{itemize}
	{\bf Milestone: Submit dissertation} 
\end{enumerate}



 

