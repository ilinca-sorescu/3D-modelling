
% Draft #1 (final?)

\vfil

\centerline{\Large Computer Science Project Proposal}
\vspace{0.4in}
\centerline{\Large Constructing 3D models from image sequences}
\vspace{0.4in}
\centerline{\large I. G. Sorescu, Newnham College}
\vspace{0.3in}
\centerline{\large Originator: I. G. Sorescu}
\vspace{0.3in}
\centerline{\large 17$^{th}$ October 2014}

\vfil


\noindent
{\bf Project Supervisor:} K. Palyutina
\vspace{0.2in}

\noindent
{\bf Director of Studies:} Dr J. K. Fawcett
\vspace{0.2in}
\noindent
 
\noindent
{\bf Project Overseers:} Dr~A.~Rice  \& Dr~T.~G.~Griffin


% Main document

\section*{Introduction and Description of the Work}

The aim of this project is to extract a 3D model of a small object from a sequence of plain images and then to export the model into a 3D geometry format such as PLY. The resulting file can then be rendered and processed by MeshLab (or similar).  \\
The main focus of the project is to obtain a wireframe model from calibrated images. A digital renderer will be used in order to provide calibrated, error-free input. The project is designed to be later extended to support real camera input.\\
(If real camera input is supported: ) The inputted set of pictures must be obtained using the following technique [add an image to represent this]: 
{\emph Define an "origin" on a perfectly horizontal table and place the camera there facing a white background. Place an object of max height X at a distance Y from the origin. Take the first picture. Rotate the object 30 degrees (counter clockwise as seen from camera) and take the second picture ... until 12 pictures are taken. Update pictures in the order in which they were taken. Assume well-lit room and no major light differences between pictures in the same set.}\\  The digital renderer will imitate this technique and output the equivalent 12 pictures.

\section*{Starting Point}
Some experience with C/C++.
IB maths and graphics.

\section*{Resources Required}
my own computer + backup plans.\\
Open source libraries I might use: OpenCV, OpenGL (for the renderer), MeshLab (to view the output) \\
For the extension: a digital camera. Specific details to be added later but anything which supports manual settings should do.  


\section*{Work to be done}
For core bits (strictly sufficient to satisfy the success criterion):
\begin{itemize}
\item write digital renderer to provide calibrated input for the modules to follow - this module will be later used to evaluate the results by adding a measured error to the inputs and comparing the outputs
\item computing the geometric relation between neighbouring images
\item compute a dense set of correspondences between neighbouring images
{\emph stereo rectification, stereo matching, dense depth map}
\item reconstructing the 3D object shape (a wireframe)
{\emph overlying a 2D triangular mesh on top of one of the images to bild corresponding 3D mesh by wrapping vertices of the triangles in 3D space by using depth maps OR volumetric depth map integration, Kalman filters}
\end{itemize}
For the extensions:
\begin{itemize}
\item add a calibration module in order to support for real camera input (as opposed to digital snapshots generated by the renderer).  
\end{itemize}

\section*{Success Criterion}
The project is considered to be a success if it can correctly reproduce the shape of an object from 12 error-free digital snapshots produced as specified in the introduction.\\ 

 Might be helpful to compare results to those generated by ARC3D.
 
\section*{Possible Extensions}
\begin{itemize}
\item support for real camera input
\item add shadowing to model
\item add texture to model
\item self-calibration: this would allow the users to input virtually any set of pictures representing an object as long as they have a large overlap and completely determine the object (i.e.: no need to take exactly 12 pictures 30 degrees apart).
\end{itemize}

\section*{Timetable}

Planned starting date is 09/10/2014.

\begin{enumerate}

\item {\bf Slot 0: 9$^{th}$ Oct - 24$^{th}$ Oct}
	\begin{itemize}
		\item Read relevant literature and plan the project.
	\end{itemize}
	{\bf Milestone:} Submit proposal
\item {\bf Slot 1: 25$^{th}$ Oct - 14$^{th}$ Nov}
	\begin{itemize}
		\item Further research: gain a deeper understanding of the techniques involved in edge detection and 3D modelling
	\end{itemize}
	{\bf Milestone: Have a clear understanding of the techniques needed to complete the project} 
\item {\bf Slot 2: 15$^{th}$ Nov  - 28$^{th}$ Nov}
	\begin{itemize}
		\item Start implementation: implement the digital renderer
		\item Use the renderer to design a few basic unit tests for the edge detector, dense set generator and wireframe generator.
	\end{itemize}
	{\bf Milestone: finish implementing the digital renderer and the test harness}
\item {\bf Slot 3: 28$^{th}$ Nov  - 5$^{th}$ Dec}
\begin{itemize}
		\item Implement the wireframe generator. This should be implemented first because it is the riskiest module of the project. 
	\end{itemize}
	{\bf Milestone: Pass the unit tests for the wireframe generator} 
\item {\bf Slot 4: 6$^{th}$ Dec -  26$^{th}$ Dec}
	\begin{itemize}
		\item Implement dense set generator
	\end{itemize}
	{\bf Milestone: Pass the unit tests for the dense set generation module} 
\item {\bf Slot 5: 27$^{th}$ Dec - 16$^{th}$ Jan}
\begin{itemize}
		\item Implement the edge-detector
	\end{itemize}
	{\bf Milestone: Finish implementation, Pass the unit tests for the edge-detection module} 
\item {\bf Slot 6: 17$^{th}$ Jan  - 30$^{th}$ Jan}
	\begin{itemize}
		\item Buffer time: catch up or start doing the extensions
		\item Write progress report
	\end{itemize}
	{\bf Milestone: Submit progress report} 
\item {\bf Slot 7: 31$^{st}$ Jan - 13 $^{th}$ Feb}
	\begin{itemize}
		\item Further tests
	\end{itemize}
	{\bf Milestone: Finish writing integration tests and more in-depth unit tests} 
\item {\bf Slot 8: 13$^{st}$ Feb - 27 $^{th}$ Feb}
	\begin{itemize}
		\item Debug
	\end{itemize}
	{\bf Milestone: Pass all of the tests}
\item {\bf Slot 9: 27$^{st}$ Feb - 13$^{th}$ Mar}
	\begin{itemize}
		\item Catch-up time or extensions
	\end{itemize}
	{\bf Milestone: Pass all of the tests} 
\item {\bf Slot 10: 14$^{th}$ Mar - 3$^{rd}$ Apr}
	\begin{itemize}
		\item Analysis
	\end{itemize}
	{\bf Milestone: Finish doing the evaluation graphs} 
\item {\bf Slot 11: 4$^{th}$ Apr - 17$^{th}$ Apr}
	\begin{itemize}
		\item Plan and start writing the dissertation
	\end{itemize}
	{\bf Milestone: Write the main parts of the dissertation}
\item {\bf Slot 12: 18$^{th}$ Apr - 8$^{th}$ May}
	\begin{itemize}
		\item Write the dissertation
	\end{itemize}
	{\bf Milestone: Finish writing dissertation}	 
\item {\bf Slot 13: 9$^{th}$May  - 15$^{th}$May}
	\begin{itemize}
		\item Safety slot
	\end{itemize}
	{\bf Milestone: Submit dissertation} 
\end{enumerate}



 

